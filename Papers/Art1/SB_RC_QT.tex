% ****** Start of file apssamp.tex ******
%
%   This file is part of the APS files in the REVTeX 4.2 distribution.
%   Version 4.2a of REVTeX, December 2014
%
%   Copyright (c) 2014 The American Physical Society.
%
%   See the REVTeX 4 README file for restrictions and more information.
%
% TeX'ing this file requires that you have AMS-LaTeX 2.0 installed
% as well as the rest of the prerequisites for REVTeX 4.2
%
% See the REVTeX 4 README file
% It also requires running BibTeX. The commands are as follows:
%
%  1)  latex apssamp.tex
%  2)  bibtex apssamp
%  3)  latex apssamp.tex
%  4)  latex apssamp.tex
%
\documentclass[%
%reprint,
%superscriptaddress,
%groupedaddress,
%unsortedaddress,
%runinaddress,
%frontmatterverbose, 
preprint,
%preprintnumbers,
onecolumn,
notitlepag,
%nofootinbib,
%nobibnotes,
%bibnotes,
 amsmath,amssymb,
 aps,
 pra,
%prb,
%rmp,
%prstab,
%prstper,
%floatfix,
]{revtex4-2}

\usepackage{graphicx}% Include figure files
\usepackage{dcolumn}% Align table columns on decimal point
\usepackage{bm}% bold math
\usepackage{bbm}%bold symbols
\usepackage{hyperref}% add hypertext capabilities
\usepackage[tickmarkheight=0.1cm]{todonotes}
\usepackage{soul}
\usepackage{url}
\usepackage{cleveref}
\usepackage{natbib}
\usepackage{showkeys}% shows citation keys	
\usepackage[draft,inline,nomargin]{fixme} 
\fxsetup{theme=color} %Para hacer comentarios y resaltarlos con colores.
\FXRegisterAuthor{cv}{acv}{CV}

%%% MACROS 

\newcommand{\be}{\begin{equation}}
\newcommand{\ee}{\end{equation}}
\newcommand{\ben}{\begin{equation*}}
\newcommand{\een}{\end{equation*}}

\newcommand{\tr}{\mbox{Tr}} 
\providecommand{\abs}[1]{\lvert#1\rvert}                                    
\newcommand{\bra}[1]{\ensuremath{\langle #1 |}}
\newcommand{\ket}[1]{\ensuremath{| #1 \rangle}}
\newcommand{\prj}[1]{\ensuremath{| #1 \rangle \langle #1 |}}
\newcommand{\ovl}[2]{\ensuremath{\langle #1 | #2 \rangle}}
\newcommand{\matel}[3]{\ensuremath{\langle #1 | #2 | #3 \rangle}}
\newcommand{\expval}[2]{\ensuremath{\langle{#1}\rangle_{#2}}}


\begin{document}

%\preprint{}

\title[]{Reaction coordinate and quantum trjectories}

\author{Luis Eduardo Herrera}
\email{leherrerar@unal.edu.co }
\affiliation{Departamento de Física, Universidad Nacional de Colombia, Carrera 30 No.45-03, Bogotá, Colombia.}
\author{Carlos Viviescas}
\email{clviviescasr@unal.edu.co }
\affiliation{Departamento de Física, Universidad Nacional de Colombia, Carrera 30 No.45-03, Bogotá, Colombia.}

\begin{abstract}
\todo[inline]{Write the abstract at the end}
\end{abstract}
	
\maketitle
%------------------------------------------------------------
%
\section{Introduction}

\todo[inline]{Write the introduction at the end}

\section{Spin-boson model}

In the first part of the this notes we closely follows Gernot \todo[]{Add a reference to Gernots notes on quantum transport.}

We consider a single two-level system (TLS) linearly coupled to a bosonic environment, 
\begin{equation}
H = \frac{\omega}{2} \sigma_{z}+\sum_{k} \frac{t_{k}^{2}}{\omega_{k}} S^{2}+S \sum_{k} t_{k}\left(a_{k}+a_{k}^{\dagger}\right)+\sum_{k} \omega_{k} a_{k}^{\dagger} a_{k}.
\end{equation}
Here, $a_{k}$ and $a^{\dagger}_{k}$ are bosonic anihilation and creation operators of the environment mode $k$, with frequency $\omega_{k}$, satisfying $[b_{k},b^{\dagger}_{k'}] = \delta_{kk'} \mathbbm{1}$, and $S$ is the coupling operator acting on the system and accounting for its interaction with all modes of the envirionment through coupling amplitudes $t_{k}$, which without loss of generality are chosen to be real. The environment spectral density is then $J^{(0)} = \sum_{k} t_{k}^2 \delta (\omega -  \omega_{k})$.

In this notes we shall consider the two cases $S = \sigma_{z}$, the pure-dephasing limit, and $S = \sigma_{x}$, the dissipative spin-boson model. For both models $S^2 =1$ and the renormalization becomes a trivial shift.

\subsection{Reaction coordinate mapping}

We map the system \todo[]{Talk about the Bogoliubov transforms} into a model where a collective mode of the environment, the reaction coordinate (RC), denoted by the creation (annihilation) operator $b$ ($b^{\dag}$), is incorporated into an effective system Hamiltonian coupled to a residual environment \todo[]{Luis: please write the appendix explaining how to get the reaction coordinate}(see Appendix~\ref{A:RCmap}). The spin-boson hamiltonian decomposes into
\begin{equation}
H = H_{S} + H_{B} + H_{I},
\end{equation}
with system Hamiltonian
\begin{equation}
H_{\text{S}}=\frac{\omega}{2} \sigma^{z}+\Omega_{0}\left(b^{\dagger}+\frac{g}{\Omega_{0}} S\right)\left(b+\frac{g}{\Omega_{0}} S\right)+ \Omega_{0} \Delta (b + b^{\dag})^{2},
\end{equation}
where the coupling strength and the energy of the RC are obtained via
\begin{equation}
g^{2}=\frac{1}{2 \pi \Omega_{0}} \int_{0}^{\infty} \omega J^{(0)}(\omega) d \omega, 
\quad
\Omega_{0}^{2}=\frac{\int_{0}^{\infty} \omega J^{(0)}(\omega) d \omega}{\int_{0}^{\infty} \frac{J^{(0)}(\omega)}{\omega} d \omega},
\end{equation} 
and the system renormalization is given by
\begin{equation}
\Omega_{0} \cdot \Delta \equiv \sum_{k} \frac{h_{k}^{2}}{\Omega_{k}}=\frac{1}{2 \pi} \int_{0}^{\infty} \frac{J^{(0)}(\omega)}{\omega} d \omega.
\end{equation}
The residual environment, with creation (annihilation) operators $b_{k}$ ($b_{k}^{\dag}$),  has hamiltonian
\begin{equation}
H_{B}= \sum_{k} \Omega_{k}\ b_{k}^{\dagger} b_{k}, 
\end{equation}
and is characterized by its spectral density
\begin{equation}
J^{(1)}(\omega) \equiv 2 \pi \sum_{k} h_k^2 \delta(\omega-\Omega_k) = \frac{4 g^{2} J^{(0)}(\omega)}{\left[\frac{1}{\pi} \mathcal{P} \int \frac{J^{(0)}\left(\omega^{\prime}\right)}{\omega-\omega^{\prime}} d \omega^{\prime}\right]^{2}+\left[J^{(0)}(\omega)\right]^{2}}.
\end{equation}
It couples only to the RC through the interaction hamiltonian
\begin{equation}
H_{I} = (b + b^{\dag}) \sum_{k} h_{k} (b_{k} + b_{k}^{\dagger}). 
\end{equation}

\subsection{Lindblad master equation and quantum trajectories}

We now assume the environment is only weakly coupled to the residual environment, and threat it within a Lindblad master equation for the reduced state of the composite TLS and RC system (see Appendix~\ref{B:LME}),
\begin{equation}
\label{eq:LME}
\dot{\rho}(t) = -i [H_{S}, \rho(t)] + \mathcal{D}[\sqrt{\gamma} b] \rho + \mathcal{D}[\sqrt{\bar{\gamma}} b^{\dag}] \rho \equiv \mathcal{L}\rho,
\end{equation}
where $\mathcal{D}[J]\rho \equiv J\rho J^{\dag} - \frac{1}{2} J^{\dag} J \rho - \frac{1}{2}  \rho J^{\dag} J \rho$, and we assumed that only the residual environment is in thermal equilibrium during the whole evolution. 

Instead of directly solving \eqref{eq:LME} we consider particular diffusive unravelings of it. Assuming a continuous monitoring of the residual environment with perfectly efficiency, the state of the system will evolve conditioned on the measurement record according to the diffusive stochastic master equation
\begin{equation}
d\rho_{c} = \mathcal{L}\rho_{c} \, dt +  \mathcal{H}[d\xi_{1}\,\sqrt{\gamma} b] \rho_{c} +  \mathcal{H}[d\xi_{2}\, \sqrt{\bar{\gamma}} b^{\dag}] \rho_{c},
\end{equation} 
where $\mathcal{H}[J] \rho \equiv  J \rho + \rho J^{\dag} - \tr [J \rho + \rho J^{\dag}] \rho$.  The complex Wiener process $d\boldsymbol{\xi} = (d\xi_{1}, d\xi_{2})^{\top}$ has vanishing ensemble average, $E[d\boldsymbol{\xi}] = 0$, with correlations 
\begin{equation}
d\boldsymbol{\xi} d\boldsymbol{\xi}^{\dag} = \mathbbm{1} dt, \quad  d\boldsymbol{\xi} d\boldsymbol{\xi}^{\top} = \mathsf{Y} dt.
\end{equation}
Here $\mathbbm{1}$ is the identity matrix and $\mathsf{Y}$ is a $2 \times 2$ complex symmetric matrix. Physical choices of $\mathsf{Y}$ are restricted by the condition that the $4 \times 4$ correlation matrix $U$ for the real vector $(\text{Re}\, d\boldsymbol{\xi} ,\text{Im} d\boldsymbol{\xi} )^{\top}$,
\be
\mathsf{U} = \frac{1}{2} \begin{pmatrix} \mathbbm{1} + \mathrm{Re}\mathsf{Y} & \mathrm{Im}\mathsf{Y}  \\ \mathrm{Im}\mathsf{Y} & \mathbbm{1} - \mathrm{Re}\mathsf{Y}  \end{pmatrix},
\ee
is positive definite. Associated with each unraveling is the measurement record upon which the evolution of $\rho_{c}$ is conditioned,  represented by the  the complex currents
\begin{equation}
\mathbf{y} \, dt  = 
\begin{pmatrix}
\langle \sqrt{\gamma}b + \mathsf{Y}_{11} \sqrt{\gamma}b^{\dag} +  \mathsf{Y}_{12} \sqrt{\bar{\gamma}} b \rangle \\
\langle \sqrt{\bar{\gamma}}b^{\dag} + \mathsf{Y}_{21} \sqrt{\gamma}b^{\dag} +  \mathsf{Y}_{22} \sqrt{\bar{\gamma}} b \rangle
\end{pmatrix} dt +
 d\boldsymbol{\xi}
\end{equation}
where each component represents a specific detection event.

We now turn to the application of our ideas to two specific cases.

\section{Applications}

We now turn to the application of our ideas to two specific cases. 

\cite{IlesSmith:2014db}

\subsection{Pure dephasing}

In this section we consider 

Let us take a two-level system, or qubit, described by the Pauli spin operator σz, and model the environment as a collection of bosonic field modes. In practice, such fields can yield an appropriate effective description even if the actual environment looks quite differently, in particular if the environmental coupling is a sum of many small contributions.2 What is fairly non-generic in the present model is the type of coupling between system and environment, which is taken to commute with the system Hamiltonian.


\appendix
\section{The reaction coordinate mapping}
\label{A:RCmap}

\section{RC Lindblad master equation}
\label{B:LME}

%
\bibliographystyle{plain}
\bibliography{SB_RC_QT_bib}
%------------------------------------------------------------
%
\end{document}

